\documentclass[12pt]{article}
\usepackage{amsmath}
\usepackage{latexsym}
\usepackage{amsfonts}
\usepackage[normalem]{ulem}
\usepackage{soul}
\usepackage{array}
\usepackage{amssymb}
\usepackage{extarrows}
\usepackage{graphicx}
\usepackage[backend=biber,
style=numeric,
sorting=none,
isbn=false,
doi=false,
url=false,
]{biblatex}\addbibresource{bibliography.bib}

\usepackage{subfig}
\usepackage{wrapfig}
\usepackage{txfonts}
\usepackage{wasysym}
\usepackage{enumitem}
\usepackage{adjustbox}
\usepackage{ragged2e}
\usepackage[svgnames,table]{xcolor}
\usepackage{tikz}
\usepackage{longtable}
\usepackage{changepage}
\usepackage{setspace}
\usepackage{hhline}
\usepackage{multicol}
\usepackage{tabto}
\usepackage{float}
\usepackage{multirow}
\usepackage{makecell}
\usepackage{fancyhdr}
\usepackage[toc,page]{appendix}
\usepackage[hidelinks]{hyperref}
\usetikzlibrary{shapes.symbols,shapes.geometric,shadows,arrows.meta}
\tikzset{>={Latex[width=1.5mm,length=2mm]}}
\usepackage{flowchart}\usepackage[paperheight=11.69in,paperwidth=8.27in,left=1.18in,right=0.39in,top=0.79in,bottom=0.79in,headheight=1in]{geometry}
\usepackage[utf8]{inputenc}
\usepackage[russian]{babel}
\usepackage[T1,T2A]{fontenc}
\TabPositions{0.5in,1.0in,1.5in,2.0in,2.5in,3.0in,3.5in,4.0in,4.5in,5.0in,5.5in,6.0in,6.5in,}

\urlstyle{same}


 %%%%%%%%%%%%  Set Depths for Sections  %%%%%%%%%%%%%%

% 1) Section
% 1.1) SubSection
% 1.1.1) SubSubSection
% 1.1.1.1) Paragraph
% 1.1.1.1.1) Subparagraph


\setcounter{tocdepth}{5}
\setcounter{secnumdepth}{5}


 %%%%%%%%%%%%  Set Depths for Nested Lists created by \begin{enumerate}  %%%%%%%%%%%%%%


\setlistdepth{9}
\renewlist{enumerate}{enumerate}{9}
		\setlist[enumerate,1]{label=\arabic*)}
		\setlist[enumerate,2]{label=\alph*)}
		\setlist[enumerate,3]{label=(\roman*)}
		\setlist[enumerate,4]{label=(\arabic*)}
		\setlist[enumerate,5]{label=(\Alph*)}
		\setlist[enumerate,6]{label=(\Roman*)}
		\setlist[enumerate,7]{label=\arabic*}
		\setlist[enumerate,8]{label=\alph*}
		\setlist[enumerate,9]{label=\roman*}

\renewlist{itemize}{itemize}{9}
		\setlist[itemize]{label=$\cdot$}
		\setlist[itemize,1]{label=\textbullet}
		\setlist[itemize,2]{label=$\circ$}
		\setlist[itemize,3]{label=$\ast$}
		\setlist[itemize,4]{label=$\dagger$}
		\setlist[itemize,5]{label=$\triangleright$}
		\setlist[itemize,6]{label=$\bigstar$}
		\setlist[itemize,7]{label=$\blacklozenge$}
		\setlist[itemize,8]{label=$\prime$}

\pagenumbering{gobble}
\setlength{\topsep}{0pt}\setlength{\parskip}{9.96pt}
\setlength{\parindent}{0pt}

 %%%%%%%%%%%%  This sets linespacing (verticle gap between Lines) Default=1 %%%%%%%%%%%%%%


\renewcommand{\arraystretch}{1.3}


%%%%%%%%%%%%%%%%%%%% Document code starts here %%%%%%%%%%%%%%%%%%%%



\begin{document}
\begin{titlepage}
\begin{Center}
ГУАП
\end{Center}\par

\begin{Center}
КАФЕДРА № 43
\end{Center}\par
\vspace{\baselineskip}
ОТЧЕТ \\
ЗАЩИЩЕН С ОЦЕНКОЙ\par

ПРЕПОДАВАТЕЛЬ\par



%%%%%%%%%%%%%%%%%%%% Table No: 1 starts here %%%%%%%%%%%%%%%%%%%%


\begin{table}[H]
 			\centering
\begin{tabular}{p{2.05in}p{0.0in}p{1.76in}p{-0.01in}p{1.89in}}
%row no:1
\multicolumn{1}{p{2.05in}}{\Centering {ст. преп.}} & 
\multicolumn{1}{p{0.0in}}{} & 
\multicolumn{1}{p{1.76in}}{} & 
\multicolumn{1}{p{-0.01in}}{} & 
\multicolumn{1}{p{1.89in}}{\Centering {М.Д. Поляк}} \\
\hhline{-~-~-}
%row no:2
\multicolumn{1}{p{2.05in}} {\Centering{\fontsize{10pt}{12.0pt}\selectfont  {должность, уч. степень, звание}}} & 
\multicolumn{1}{p{0.0in}}{} & 
\multicolumn{1}{p{1.76in}} {\Centering{\fontsize{10pt}{12.0pt}\selectfont  {подпись, дата}}} & 
\multicolumn{1}{p{-0.01in}}{} & 
\multicolumn{1}{p{1.89in}} {\Centering{\fontsize{10pt}{12.0pt}\selectfont  {инициалы, фамилия}}} \\
\hhline{~~~~~}

\end{tabular}
 \end{table}


%%%%%%%%%%%%%%%%%%%% Table No: 1 ends here %%%%%%%%%%%%%%%%%%%%


\vspace{\baselineskip}


%%%%%%%%%%%%%%%%%%%% Table No: 2 starts here %%%%%%%%%%%%%%%%%%%%


\begin{table}[H]
 			\centering
\begin{tabular}{p{6.49in}}
%row no:1
\multicolumn{1}{p{6.49in}}{\fontsize{14pt}{16.8pt}\selectfont {\section*{\Centering {ОТЧЕТ О ЛАБОРАТОРНОЙ РАБОТЕ №1}}}}\\
\hhline{~}
%row no:2
\multicolumn{1}{p{6.49in}}{\section*{\Centering {СОЗДАНИЕ ПОЛЬЗОВАТЕЛЬСКИХ ФУНКЦИЙ В ПРИЛОЖЕНИИ EXCEL}}
} \\
\hhline{~}
%row no:3
\multicolumn{1}{p{6.49in}}{\subsubsection*{\Centering {по курсу: ИНФОРМАЦИОННЫЕ ТЕХНОЛОГИИ}}
} \\
\hhline{~}
%row no:4
\multicolumn{1}{p{6.49in}}{} \\
\hhline{~}
%row no:5
\multicolumn{1}{p{6.49in}}{} \\
\hhline{~}

\end{tabular}
 \end{table}


%%%%%%%%%%%%%%%%%%%% Table No: 2 ends here %%%%%%%%%%%%%%%%%%%%

РАБОТУ ВЫПОЛНИЛ\par



%%%%%%%%%%%%%%%%%%%% Table No: 3 starts here %%%%%%%%%%%%%%%%%%%%


\begin{table}[H]
 			\centering
\begin{tabular}{p{1.3in}p{1.0in}p{-0.04in}p{1.63in}p{-0.04in}p{1.63in}}
%row no:1
\multicolumn{1}{p{1.3in}}{СТУДЕНТ ГР. №} & 
\multicolumn{1}{p{1.0in}}{\Centering {5131}} & 
\multicolumn{1}{p{-0.04in}}{} & 
\multicolumn{1}{p{1.63in}}{} & 
\multicolumn{1}{p{-0.04in}}{} & 
\multicolumn{1}{p{1.63in}}{\Centering {А.Ю. Антонов}} \\
\hhline{~-~-~-}
%row no:2
\multicolumn{1}{p{1.3in}}{} & 
\multicolumn{1}{p{1.0in}}{} & 
\multicolumn{1}{p{-0.04in}}{} & 
\multicolumn{1}{p{1.63in}} {\Centering{\fontsize{10pt}{12.0pt}\selectfont  {подпись, дата}}} & 
\multicolumn{1}{p{-0.04in}}{} & 
\multicolumn{1}{p{1.63in}} {\Centering{\fontsize{10pt}{12.0pt}\selectfont  {инициалы, фамилия}}} \\
\hhline{~~~~~~}

\end{tabular}
 \end{table}


%%%%%%%%%%%%%%%%%%%% Table No: 3 ends here %%%%%%%%%%%%%%%%%%%%


\vspace{\baselineskip}
\vspace{\baselineskip}
\vspace{\baselineskip}
\vspace{\baselineskip}
\vspace{\baselineskip}
\begin{Center}
Санкт-Петербург \the\year{}
\end{Center}\par
\end{titlepage}

\section*{1. Цель работы}
Получение навыков по расчету временных параметров сетевых моделей, представленных в виде матрицы.

\section*{2. Вариант задания}
Произвести расчет временных параметров сетевой модели матричным методом в соответствии с вариантом, представленным в приложении.

\begin{enumerate}
    \item Проанализировать вариант задания.
    \item Построить сетевой график.
    \item Произвести нумерацию вершин сетевого графика.
    \item Сформировать матричное представление сетевого графика.
    \item Произвести расчет временных параметров матричным методом.
    \item Оформить отчет.
    \item Защитить отчет.
\end{enumerate}

\textbf{Вариант 1.}

\section*{3. Сетевой график}
\begin{figure}[H]
    \centering
    \includegraphics[width=0.8\textwidth]{aoa_graph.png}
    \caption{Сетевой график AoA}
\end{figure}

\begin{figure}[H]
    \centering
    \includegraphics[width=0.8\textwidth]{aon_graph.png}
    \caption{Сетевой график AoN}
\end{figure}

\section*{4. Матричное представление сетевого графика}
\begin{table}[H]
    \centering
    \begin{tabular}{|c|c|c|c|c|}
        \hline
        & A & B & C & \ldots \\
        \hline
        A & 0 & 3 & 0 & \ldots \\
        \hline
        B & 0 & 0 & 5 & \ldots \\
        \hline
        \ldots & \ldots & \ldots & \ldots & \ldots \\
        \hline
    \end{tabular}
    \caption{Матричное представление сетевого графа}
\end{table}

\section*{5. Расчет временных параметров матричным методом}
\begin{table}[H]
    \centering
    \begin{tabular}{|c|c|c|c|c|c|}
        \hline
        Работа & t & ES & EF & LS & LF \\
        \hline
        I-A & 2 & 0 & 2 & 0 & 2 \\
        \hline
        A-E & 3 & 2 & 5 & 2 & 5 \\
        \hline
        \ldots & \ldots & \ldots & \ldots & \ldots & \ldots \\
        \hline
    \end{tabular}
    \caption{Расчет временных параметров матричным методом}
\end{table}

\section*{6. Результаты расчетов}
\begin{table}[H]
    \centering
    \begin{tabular}{|c|c|c|c|c|}
        \hline
        Событие & T\_ран & T\_поз & R & Крит. путь \\
        \hline
        1 & 0 & 0 & 0 & Да \\
        \hline
        2 & 2 & 2 & 0 & Да \\
        \hline
        \ldots & \ldots & \ldots & \ldots & \ldots \\
        \hline
    \end{tabular}
    \caption{Расчет временных параметров событий}
\end{table}

\begin{table}[H]
    \centering
    \begin{tabular}{|c|c|c|}
        \hline
        Работа & R\_полн & R\_своб \\
        \hline
        I-A & 0 & 0 \\
        \hline
        A-E & 0 & 0 \\
        \hline
        \ldots & \ldots & \ldots \\
        \hline
    \end{tabular}
    \caption{Расчет резервов времени работ}
\end{table}

\textbf{Критический путь: (I,A)(A,E)(E,M)(M,K)(K,C)} \\
\textbf{Продолжительность критического пути: 14}

\section*{7. Выводы}
В результате работы были получены навыки по расчету временных параметров сетевых моделей, представленных в виде матрицы. Были найдены временные параметры, критический путь и его продолжительность.

\end{document}