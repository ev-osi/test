\documentclass[a4paper,12pt]{article}
\usepackage[utf8]{inputenc}
\usepackage[russian]{babel}
\usepackage{geometry}
\usepackage{array}
\usepackage{multirow}
\usepackage{graphicx}
\usepackage{float}
\usepackage{setspace}

\geometry{left=2cm,right=1.5cm,top=2cm,bottom=2cm}
\setlength{\parindent}{0pt}

\renewcommand{\arraystretch}{1.2}
\newcolumntype{C}[1]{>{\centering\arraybackslash}p{#1}}

\begin{document}

% Титульная страница
\begin{titlepage}
    \centering
    \vspace*{1cm}
    
    \textbf{\large ГУАП} \\
    \textbf{\large КАФЕДРА № 42} \\
    \vspace{2cm}
    
    \textbf{\Large ОТЧЕТ} \\
    \textbf{\Large ЗАЩИЩЕН С ОЦЕНКОЙ} \\
    \vspace{2cm}
    
    \begin{tabular}{|C{5cm}|C{2cm}|C{5cm}|}
        \hline
        \multirow{2}{5cm}{\centering канд. техн. наук., доцент} & \multirow{2}{2cm}{} & \multirow{2}{5cm}{} \\
        & & \\ \hline
        должность, уч. степень, звание & подпись, дата & инициалы, фамилия \\
        & & Д. В. Богданов \\
        \hline
    \end{tabular}
    \vspace{2cm}
    
    \begin{tabular}{|c|}
        \hline
        \textbf{ОТЧЕТ О ЛАБОРАТОРНОЙ РАБОТЕ №3} \\
        \hline
        \textbf{Расчет параметров сетевого графика матричным методом} \\
        \hline
        \textbf{по курсу: УПРАВЛЕНИЕ ПРОГРАММНЫМИ ПРОЕКТАМИ} \\
        \hline
        \\
        \hline
    \end{tabular}
    \vspace{2cm}
    
    \begin{tabular}{|C{3cm}|C{3cm}|C{2cm}|C{5cm}|}
        \hline
        \textbf{СТУДЕНТ гр. №} & \textbf{4233K} & \textbf{07.05.2025} & \textbf{} \\
        \hline
        & & подпись, дата & инициалы, фамилия \\
        & & & Е.А. Осинкин \\
        \hline
    \end{tabular}
    \vfill
    
    \textbf{\large Санкт-Петербург 2025}
\end{titlepage}

\section*{1. Цель работы}
Получение навыков по расчету временных параметров сетевых моделей, представленных в виде матрицы.

\section*{2. Вариант задания}
Произвести расчет временных параметров сетевой модели матричным методом в соответствии с вариантом, представленным в приложении.

\begin{enumerate}
    \item Проанализировать вариант задания.
    \item Построить сетевой график.
    \item Произвести нумерацию вершин сетевого графика.
    \item Сформировать матричное представление сетевого графика.
    \item Произвести расчет временных параметров матричным методом.
    \item Оформить отчет.
    \item Защитить отчет.
\end{enumerate}

\textbf{Вариант 1.}

\section*{3. Сетевой график}
\begin{figure}[H]
    \centering
    \includegraphics[width=0.8\textwidth]{aoa_graph.png}
    \caption{Сетевой график AoA}
\end{figure}

\begin{figure}[H]
    \centering
    \includegraphics[width=0.8\textwidth]{aon_graph.png}
    \caption{Сетевой график AoN}
\end{figure}

\section*{4. Матричное представление сетевого графика}
\begin{table}[H]
    \centering
    \begin{tabular}{|c|c|c|c|c|}
        \hline
        & A & B & C & \ldots \\
        \hline
        A & 0 & 3 & 0 & \ldots \\
        \hline
        B & 0 & 0 & 5 & \ldots \\
        \hline
        \ldots & \ldots & \ldots & \ldots & \ldots \\
        \hline
    \end{tabular}
    \caption{Матричное представление сетевого графа}
\end{table}

\section*{5. Расчет временных параметров матричным методом}
\begin{table}[H]
    \centering
    \begin{tabular}{|c|c|c|c|c|c|}
        \hline
        Работа & t & ES & EF & LS & LF \\
        \hline
        I-A & 2 & 0 & 2 & 0 & 2 \\
        \hline
        A-E & 3 & 2 & 5 & 2 & 5 \\
        \hline
        \ldots & \ldots & \ldots & \ldots & \ldots & \ldots \\
        \hline
    \end{tabular}
    \caption{Расчет временных параметров матричным методом}
\end{table}

\section*{6. Результаты расчетов}
\begin{table}[H]
    \centering
    \begin{tabular}{|c|c|c|c|c|}
        \hline
        Событие & T\_ран & T\_поз & R & Крит. путь \\
        \hline
        1 & 0 & 0 & 0 & Да \\
        \hline
        2 & 2 & 2 & 0 & Да \\
        \hline
        \ldots & \ldots & \ldots & \ldots & \ldots \\
        \hline
    \end{tabular}
    \caption{Расчет временных параметров событий}
\end{table}

\begin{table}[H]
    \centering
    \begin{tabular}{|c|c|c|}
        \hline
        Работа & R\_полн & R\_своб \\
        \hline
        I-A & 0 & 0 \\
        \hline
        A-E & 0 & 0 \\
        \hline
        \ldots & \ldots & \ldots \\
        \hline
    \end{tabular}
    \caption{Расчет резервов времени работ}
\end{table}

\textbf{Критический путь: (I,A)(A,E)(E,M)(M,K)(K,C)} \\
\textbf{Продолжительность критического пути: 14}

\section*{7. Выводы}
В результате работы были получены навыки по расчету временных параметров сетевых моделей, представленных в виде матрицы. Были найдены временные параметры, критический путь и его продолжительность.

\end{document}